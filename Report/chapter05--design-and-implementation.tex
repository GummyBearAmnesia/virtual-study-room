\chapter{Design and implementation}
\label{chap:design-and-implementation}

\defaultInstructions

\begin{length}
There are no strict length requirements for this chapter.  However, it is important that the writing is clear and concise.  Avoid repeating yourself.  Make sure to clearly signpost evidence for claims.  It should be possible for a reader to understand this chapter without consulting other sources.  It is expected that a typical project needs 3--6 pages, but this can vary considerably from project to project.
\end{length}

\begin{expectations}
In this chapter, the team should discuss the key design and implementation decisions made during the project, and how these are reflected in the end product.  Design refers to the way the software system is organised into smaller components and the way in which they are related.  We are \emph{not} looking for a discussion of a series of screenshots of your application.

Normally, all teams should present the overall architecture of the software.  The remainder of the chapter will depend on the type of system that is built.  To decide what to cover, consider the following to aspects of the project:
\begin{itemize}
\item What design and implementation decisions did you make to improve software quality?
\item What design and implementation decisions did you make to achieve challenging functional and non-functional specifications
\end{itemize}
Depending on your answers to these questions, you may cover the design an implementation specific components, interfaces, the overall class structure, the database design, algorithms, business processes, etc.

A good report considers alternative options for key decisions and includes sound justifications for the decisions made.  If applicable, you may reflect on changes made during the project.  Ideally, key decisions are rooted in Software Engineering theory.
\end{expectations}

\section{Architecture}
\label{sect:architecture}
%\begin{expectations}
%Add other sections and sub-sections as necessary.
%\end{expectations}

Since our main stakeholders are students, we decided to base our design decisions on what would attract students to our website. Our colour palette was decided early on, and we focused on a gender neutral, yet colourful and engaging design. Simplicity was something that the entire team wanted; as students, we know how difficult it is to navigate cluttered and confusing websites. Due to this, we were encouraged to ensure that the focus remained on the actual studying and not an immense overload of things to explore. The Login and Signup pages were created to mirror each other, with a straightforward UI and clear input fields. Any errors are repeated back to the user to ensure a smooth login/signup process. After the user logs in, they are directed to the main Dashboard page, where there is a ‘mini version’ of each of the features. This way, the user gets an immediate idea of the overview of features, without getting overwhelmed. If they wish to ‘expand’ the feature, they can simply click on the relevant icon (e.g. a simplified calendar icon, where if you click it, it brings you to a dedicated calendar page with multiple interactive features you can use). Again, with the Group-Study page, we retained a simplified approach. As mentioned before, this is due to the possibility that the user may become distracted if the UI is too cluttered. This idea is reflected in our colour scheme also, with the palette containing subtle and light colours that are easy on the eyes.

\section{Implementation Details}
\label{sect:implementation-details}
1. Initial design plan: We gained inspiration from existing study group pages and the elements / components needed by the students. Brainstormed and mapped components onto paper: added components block-wise like a grid on the dashboard and group study pages. We used Figma to start designing how the UI components placement, colour scheme and overall presentation/layout of all the different pages: Login, Welcome Page, Dashboard, Group Study Room.

2. Finalized features: through the elimination process, we nailed down the ideas that reflected the goals of our project and made our idea unique.
\begin{itemize}
    \item To-Do List (shared/individual): This allows the user to create individual lists for themselves and separate group lists (when working as a group).
    \item Shared Materials: allows multiple users, in a group study session, to be able to share materials with each other in, but this is only active whilst the session is alive.
    \item Timer: This feature can be a shared or individual timer, depending on the user’s requirements. This feature allows the user to schedule breaks within their desired focus time.
    \item Chat box: allowing users to communicate, while present in the group study room. This allows the user to be less distracted then on audio/video call, ensuring that communication is catered towards finishing the tasks.
    \item Motivational Message: These are randomly generated inspirational quotes, to encourage the user to continue studying and stay focused.
    \item Profile Box:  User can change their avatar (based on uploaded pictures from their device) and write a user description. This gives the user more creativity to customise their avatar and description to match them.
    \item Statistics : Collects data on the number of consecutive days they joined a study session (streaks) and also calculates the average time spent in the study room. This allows the user to track their time and remember to stay consistent with their studies. The user also has the option to share their data with friends.
    \item Friends List: This allows the user to see the: friends list (which users they have accepted as friends), pending requests from new users, accepted friend requests and search for different friends.
    \item Calender: This allows the user to schedule events, classes, exams and other assignment deadlines to keep up with and clearly organise a busy schedule.
    \item Badges: This is a type of reward system, that encourages the student to complete certain goals in order to achieve them e.g. completing a certain number of hours, number of tasks etc.
    \item Music: Background music helps in productivity and makes the user feel more at ease when studying. It also helps avoid distraction and boosts motivation, each user independently listens to a track of their choice. Using Spotify API, premium users can control playback, helping to limit phone distraction.
    \item Joined Users: This allows the participants who are already in the room, to see who has joined.
    \item Create/Join Room: This allows user to create a study room/join a study room. This is the foundation of the project, which creates a platform for users to share learning materials, to-do list tasks, chat with each other, independently control timer and  music features.  And be inspired by the motivational messages.

\end{itemize}

The Welcome Page/Login/Sign Up is simply designed to entice the user to login/sign up, using a vibrant pattern/colour scheme, is simplicity provides robust functionality to get the user seamlessly enter the platform.

In both the Dashboard and Group Study Page, the components are arranged in columns.
The Dashboard consists of: Profile box, To-Do List, Option to join or Create a study room, Friend Requests, Calender, Badges. These components have been added to meet requirements of the user when they first enter the website which allows them to: customise their profile, add friends on the platform, create a study room, add events and observe their yet to be obtained or obtained badges.
The Group Study Page consists of: To-Do List, Shared Materials, Music, Joined Users, Motivational Message, Timer and Chat Box. This page has been tailored to the expectation of collaborative studies, where we have included the ability to exchange study materials with all users in the group, participate in educational conversations and create shared to do lists to complete with your team. There also additional features to help avoid distraction and boost maximum productivity with Timers, Music, Motivational quotes.
The Group Study page includes a taskbar, to provide further functionalities to leave the group study room, copy-paste the room code, music feature and present the room code.

All these design decisions are reflected in the final product, the orientation of all these components has also been carefully crafted to meet the user's needs and expectations. Moreover, the layout has been slightly altered overtime to enhance user experience.



\section{Alternative Considerations}
\label{sect:alternative-considerations}
We researched and discussed many possibilities for different aspects of the project, from our tech stack to which features we should implement.

\textbf{Features:}

We had planned to use an AI API to create small quizzes using the documents uploaded to the shared materials box in the group study room; however, we had to deprioritise this feature due to lack of time, as well as us considering the other features to be of greater importance.

We also considered allowing users to join their friends' study rooms without a room code, as in having a feature that lets you see which of your friends are currently online or in a study room, and being able to join them in that study room without needing to enter the room code. After much deliberation, we decided to scrap this feature since users may not want their friends to just be able to barge into a study room without their permission, especially considering that most students have different study groups for different purposes and they may not want members of one group mixing with another.


\textbf{Tech Stack:}

We had decided on using Django for the backend and React for the frontend quite early on in the project, however, it took us a while to settle on which database to use in order to store user data. In particular, we could not choose between SQLite and Firebase.

Half of us wanted to stick to what we knew by using SQLite since that is what we used in the Small Group Project back in semester 1, whereas the other half wanted to learn something new and extend ourselves. In the end, we decided to use both. We used SQLite through the Django models for the relational part of the database such as the User table and the Friends table, and we used Firebase to store pictures and pdfs and other data that were not as relational such as the documents uploaded to the shared materials box in the group study room.

\textbf{Deployment:}

We did not want to spend too much time on deployment so we decided to stick with what we know and use PythonAnywhere to deploy Django part of the project, however we did not realise that the free version of PythonAnywhere does not support websockets. We have used websockets in multiple features of our group study room such as the chat and the shared materials box, and when deployed on PythonAnywhere, none of those features worked. We did some research and asked other groups with a similar tech stack. One of the groups told us about Render and we tried deploying our backend on Render, which was successful meaning our websockets are now functional on the deployed website.

Although we were unanimous in our decision for the backend deployment, we struggled a little in finding the correct service for our frontend deployment. We looked into Azure, Cloudflare, Heroku, and a couple other services however they all did not have a free plan that met our needs. After searching for a while, we came across Vercel and it seemed to meet all our requirements, so we decided to deploy our frontend on Vercel. It is very convenient to use Vercel since it is linked to our GitHub repository meaning it redeploys everytime main is updated which means we do not have to dedicate extra time to deploying the frontend regularly.



\section{Changes During Development}
\label{sect:changes-during-development}

HI ONESSA, THIS IS FROM THE PROJECT MANAGEMENT SECTION BUT IT EXPLAINS SOME THINGS I TALKED ABOUT. ASK THE OTHERS AND TRY TO EXPLAIN IT IN MORE DETAIL. ALSO YOU CAN TALK ABOUT THE CSS A LOT SINCE YOU DID MOST OF IT. AND ASK NATALIA TO TALK ABOTU SPOTIFY, AND PRAPTI ABOUT THE CALENDAR OK. ALSO MOVING GENERATE STUDY ROOM TO THE MIDDLE OF THE PAGE AFTER LEONARDO SAID IT WAS VERY IMPORTANT AND IT SHOULD BE IN THE MIDDLE NOT ON THE SIDE.
\begin{itemize}
    \item Issues with deployment and having to switch to vercel and render from pythonanywhere
    \item Issues with spotify premium and needing to add free music
    \item Issues with login and having to add only one login per browser to not mess with the authentication tokens
    \item Calendar taking a lot longer time than expected but becoming a much bigger and userful feature
    \item The design of the website evolving as time went on, especially after the advisors feedback
\end{itemize}