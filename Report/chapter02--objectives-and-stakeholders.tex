\chapter{Objectives and stakeholders}
\label{sect:objectives}

\begin{expectations}
This is a short chapter that identifies the objectives and stakeholders of the project.  A good report also discusses how the different types of stakeholder are affected by the project or how they could be affected by the project.

As you explain the project objectives, focus exclusively on what value the system adds to the stakeholders.  Your objectives should justify \emph{why} you are building the system.  The software requirements are \emph{not} objectives.  The software requirements explain \emph{what} your system can do.  In other words, they are a way to achieve objectives.  Software requirements do \emph{not} belong in this chapter.  Your project should have at least one objective.  Otherwise, the project is a pointless exercise.  Many objectives do not necessarily make for a better project.

Some stakeholders of a project will be obvious: e.g. the people who the beneficiaries of the value your project adds.  However, it is important to be aware of all stakeholders.  The stakeholder D.A.N.C.E. tool to identify all stakeholders and it is explained in a video on project initiation.  Consider carefully how each stakeholder is affected by your project and who the key stakeholders are.

The section headers below provide a suggested structure.  You should feel free to change it or extend it.
\end{expectations}

\section{Project objectives}
\label{sect:objectives}

\begin{itemize}
    \item \textbf{Facilitate Collaborative Learning:} Enable students to organize and participate in group study sessions effectively.
    \item \textbf{Enhance Academic Performance:} Support students in understanding complex topics through peer discussions and shared resources.
    \item \textbf{Build a Learning Community:} Foster a sense of community where students can motivate each other and learn collectively.
    \item \textbf{Fill the Gap in the Market:} Create a unique platform for seamless group study and resource sharing.
\end{itemize}

\section{Stakeholder}
\label{sect:stakeholder}

\subsection{Key stakeholders}
\begin{itemize}
    \item \textbf{Students (Primary Stakeholders):}
    \begin{itemize}
        \item \textit{Impact:} Directly benefit from improved academic performance, subject comprehension, and motivation via collaborative learning.
        \item \textit{Age Group:} Designed for students across all levels (school to university).
    \end{itemize}
    
    \item \textbf{Platform Developers and Administrators:}
    \begin{itemize}
        \item \textit{Impact:} Ensure functionality, data security, and user experience; their work dictates engagement and satisfaction.
    \end{itemize}
\end{itemize}  

\subsection{Stakeholder analysis}
As students like ourselves are some of the main stakeholders it was easy to loop in feedback from the team and our peers, which was essential for continuous improvement to the platform.

This systematic approach to identifying objectives and understanding stakeholder impact ensures that the project remains user-focused and goal-oriented.

