\chapter{Project management}
\label{chap:project-management}

\defaultInstructions

\begin{length}
There are no strict length requirements for this chapter.  However, it is important that the writing is clear and concise.  Avoid repeating yourself.  Make sure to clearly signpost evidence for claims.  It should be possible for a reader to understand this chapter without consulting other sources.  It is expected that a typical project needs 3--6 pages, but this can vary considerably from project to project.
\end{length}

\begin{expectations}
In this chapter, the team must present, reflect on, and evaluate the team's approaches to project management.  This discussion should present the main approaches to project management the team decided to use.  Include the team's rationale for these decisions, and the team's observations on the effectiveness of each approach.

During the project, the team may have made significant changes to project management approaches, or made new decisions where initial arrangements were insufficient.  A typical team will also have encountered difficulties, such as risks that materialised and impacted the team.  Discuss the most critical decisions the team made, why you made them, and what the results were.  If certain approaches were not entirely successful, discuss what you would do differently in a future project.  

A good report shows a systematic approach to project management, justifications for key decisions, and a critical reflection on the effectiveness of different approaches.  Ideally, the justifications and reflections in this chapter are based on project management theory.

Note that ``a systematic approach to project management'', ``critical reflection'', and use of ``project management theory'' do not necessary imply that your team's project management approach is a success story.  Thoughful, considered project management improves your chances of success, but projects do not always go according to plan and team's make mistakes.  The markers are looking for 
\begin{itemize}
    \item Good justifications for approaching a project in a particular way.
    \item Critical reflection on the results, including recognition of problems.
    \item The team taking ownership of the problems they encountered.
\end{itemize}

There is no one best way of organising this content.  One approach is to cover different phases of the project management lifecycle (from initialisation to project close).  Another approach is to cover different project management tasks (e.g. team management, scheduling, planning, risk, etc.).  Yet another approach is to cover the most impactful events.  Each project is different and you should take the approach that works best for your team.

There is also no minimum or maximum number of issues to cover.  Cover what is more important in the depth necessary to do the topic justice.
\end{expectations}

\section{Project Management Approaches}
\label{sect: Project Management Approaches}
\begin{length}
    Our team adopted an iterative cycle for project management, where tasks were assigned every Monday, progress was reviewed every Friday, and tasks were completed, integrated and reassigned the following Monday. This approach aligns with the principles of Agile Devlopment Process.
\end{length}

\section{Team Workflow and Decision-Making}
\label{sect: Team Workflow and Decision-Making}
\begin{length}
    We began by brainstorming the list of features we planned to include and determining which would be feasible within the timeframe. The weekly cycles, following Agile methodology, facilitated regular feedback and promoted a continuous iterative process, allowing us to adapt efficiently to challenges and changing requirements. To support this process, we used Trello, where new tasks were assigned.
\end{length}

\section{Reflection on Effectiveness}
\label{sect: Reflection on Effectiveness}
\begin{length}
    The weekly cycles facilitated regular feedback and promoted a continuous iterative process, allowing us to adapt efficiently to challenges and changing requirements. This approach ensured accountability and maintained team coordination, as progress was consistently monitored and discussed. Additionally, the regular review meetings provided opportunities for reflection and improvement, fostering a culture of continuous enhancement.

    Our choice of this approach was influenced by Agile project management theory, which emphasises adaptability, iterative progress, and stakeholder collaboration. By incorporating elements of Scrum—such as iterative cycles and progress reviews—while maintaining flexibility, we effectively balanced structure and adaptability. This systematic approach enabled us to stay aligned with project goals while remaining responsive to new insights and challenges.

    Overall, the iterative cycle proved effective in maintaining team engagement and productivity. However, a more structured Scrum framework, including defined roles (e.g., Scrum Master, Product Owner) and daily stand-ups, could potentially have enhanced communication and problem-solving efficiency. This reflection highlights the importance of adapting project management strategies to the team’s evolving needs and project requirements.

    Our website is designed to facilitate group revision and study sessions, enabling students to collaborate effectively and enhance their academic performance. The primary objective of the project is to provide a platform that fosters a supportive learning community, encouraging peer-to-peer interaction and knowledge sharing.
\end{length}